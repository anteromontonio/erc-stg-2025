\section*{Section b: Curriculum vitae and Track Record} 

\subsection*{\underline{PERSONAL DETAILS}}

\begin{itemize}
  \item \textbf{Family name, First name:} Montero, Antonio.
  \item \textbf{Researcher unique identifier(s):}
    \begin{itemize}
      \item ORCID: \href{https://orcid.org/0000-0002-3293-8517}{0000-0002-3293-8517}
      \item Google Scholar: \href{https://scholar.google.com/citations?user=GEJHkg0AAAAJ}{GEJHkg0AAAAJ}
      \item MathSciNet Author ID: \href{https://mathscinet.ams.org/mathscinet/author?authorId=1293226}{1293226}
    \end{itemize}
  \item \textbf{URL for web site:} \href{http://anteromontonio.github.io}{anteromontonio.github.io}
\end{itemize}

\subsubsection*{Education and key qualifications}

\cventry{19/08/2019}{Ph.D. (Mathematics)}{Centro de Ciencias Matematicas, National Autonomous University of Mexico (UNAM)}{Mexico}{}{Under the supervision of Dr. Daniel Pellicer, with the thesis ``Chiral extensions of toroids''.}
\cventry{2015}{M. Cs. (Mathematics)}{Centro de Ciencias Matematicas, National Autonomous University of Mexico (UNAM)}{Morelia, Mexico}{}{Thesis: ``Realizaciones regulares de poliedros en el $3$-toro (Regular realizations of polyhedra in the $3$-torus)''}
\cventry{2013}{B. Cs. (Mathematics and Physics)}{University of Michoacan}{Morelia, Mexico}{}{Thesis: ``Poliedros regulares en el $3$-toro (Regular polyhedra in the $3$-torus)''}

\subsubsection*{Current position(s)}

\cventry{2022 -- 2025}
{Znanstveni sodelavec (Research associate, postdoctoral)}%
%
{%
{Faculty of Mathematics and Physics, University of Ljubljana}%
%
}
{{Ljubljana, Slovenia}}
{}
{}
\cventry{2024 -- }
{Asistent IX-1 (Teaching assistant)}%
%
{%
{Faculty of Education, University of Ljubljana}%
%
}
{{Ljubljana, Slovenia}}
{}
{}

\subsubsection*{Previous position(s)}

\cventry{2020 -- 2022}{Becario postdoctoral (Postdoctoral fellow)}{Institute of Mathematics, National Autonomous University of Mexico (UNAM)}{Mexico City, Mexico}{}{}
\cventry{2019 -- 2020}{Postdoctoral fellow}{Department of Mathematics and Statistics, York University}{Toronto ON, Canada}{}{}

\subsection*{\underline{RESEARCH ACHIEVEMENTS AND PEER RECOGNITION}}

\subsubsection*{Research achievements}
% [Provide a list of up to ten research outputs that demonstrate how you have advanced knowledge in your field with an emphasis on more recent achievements, such as publications, articles deposited in a publicly available preprint server, books, book chapters, conference proceedings, data sets, software, patents, licenses, standards, start-up businesses or any other research outputs you deem relevant in relation to your research field and your project.

% You may include a short, factual explanation of the significance of the selected outputs, your role in producing each of them, and how they demonstrate your capacity to successfully carry out your proposed project.]

\subsubsection*{Peer recognition}

% [Provide a list of selected examples of significant recognition by your peers if applicable, such as prizes, awards, fellowships, elected academy memberships, invited presentations to major conferences or any other examples of significant recognition you deem relevant in relation to your research field and project.

% You may include a short explanation of the significance of the listed examples.]

\subsection*{\underline{ADDITIONAL INFORMATION}}


% You may provide relevant additional information on your research career to provide context to the evaluation panels when assessing your research achievements and peer recognition as described above.

\subsubsection*{Career breaks, diverse career paths and major life events}

% [You may include a short factual explanation of career breaks or diverse career paths such as secondments, volunteering, part-time work, time spent in different sectors or the effects of major life events such as long term illness as well as the effects of pandemic restrictions on research productivity.]

\subsubsection*{Other contributions to the research community} 

% [You may include a list of particularly noteworthy contributions to the research community you have made other than research achievements and peer recognition and a short explanation of these contributions. The purpose of this section is to allow the panels to take a more rounded view of your career and achievements and to ensure that any additional responsibilities, commitments and leadership roles that you have taken on beyond your individual research activities are recognised and taken into account.]


% [(for more information see ‘Information for Applicants to the Starting and Consolidator Grant 2025 Calls’)]
