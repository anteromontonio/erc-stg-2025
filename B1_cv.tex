\section*{Section b: Curriculum vitae and Track Record} 

\subsection*{\underline{PERSONAL DETAILS}}

\begin{itemize}
  \item \textbf{Family name, First name:} Montero, Antonio.
  \item \textbf{Researcher unique identifier(s):}
    \begin{itemize}
      \item ORCID: \href{https://orcid.org/0000-0002-3293-8517}{0000-0002-3293-8517}
      \item Google Scholar: \href{https://scholar.google.com/citations?user=GEJHkg0AAAAJ}{GEJHkg0AAAAJ}
      \item MathSciNet Author ID: \href{https://mathscinet.ams.org/mathscinet/author?authorId=1293226}{1293226}
    \end{itemize}
  \item \textbf{URL for web site:} \href{http://anteromontonio.github.io}{anteromontonio.github.io}
\end{itemize}

\subsubsection*{Education and key qualifications}

\cventry{19/08/2019}{Ph.D. (Mathematics)}{Centro de Ciencias Matematicas, National Autonomous University of Mexico (UNAM)}{Mexico}{}{Under the supervision of Dr. Daniel Pellicer, with the thesis ``Chiral extensions of toroids''.}
\cventry{2015}{M. Cs. (Mathematics)}{Centro de Ciencias Matematicas, National Autonomous University of Mexico (UNAM)}{Morelia, Mexico}{}{Thesis: ``Realizaciones regulares de poliedros en el $3$-toro (Regular realizations of polyhedra in the $3$-torus)''}
\cventry{2013}{B. Cs. (Mathematics and Physics)}{University of Michoacan}{Morelia, Mexico}{}{Thesis: ``Poliedros regulares en el $3$-toro (Regular polyhedra in the $3$-torus)''}

\subsubsection*{Current position(s)}

\cventry{2022 -- 2025}
{Znanstveni sodelavec (Research associate, postdoctoral)}%
%
{%
{Faculty of Mathematics and Physics, University of Ljubljana}%
%
}
{{Ljubljana, Slovenia}}
{}
{}
\cventry{2024 -- }
{Asistent IX-1 (Teaching assistant)}%
%
{%
{Faculty of Education, University of Ljubljana}%
%
}
{{Ljubljana, Slovenia}}
{}
{}

\subsubsection*{Previous position(s)}

\cventry{2020 -- 2022}{Becario postdoctoral (Postdoctoral fellow)}{Institute of Mathematics, National Autonomous University of Mexico (UNAM)}{Mexico City, Mexico}{}{}
\cventry{2019 -- 2020}{Postdoctoral fellow}{Department of Mathematics and Statistics, York University}{Toronto ON, Canada}{}{}

\subsection*{\underline{RESEARCH ACHIEVEMENTS AND PEER RECOGNITION}}

\subsubsection*{Research achievements}
\begin{enumerate}
  \item \textbf{Constructions of chiral extensions of regular toroids} \cite{MonteToled_ChiralExtensionsRegular_preprint}. In a joint work with Toledo we proved that all but infinitely many tessellations of the torus admit a chiral extension. 
  This results has a consequence the existence of infinitely many chiral polytopes in every rank higher than $4$. 
  The research in this paper is closely related to the results in \cite{CondHubORe2024_ConstructionChiralPolytopes} by Conder, Hubard and O'Reilly.
  
	\item \textbf{Combinatorial description of classical operations on polytopes} \cite{HubaMocMon2023_VoltageOperationsManiplexes,HubaMocMon_SymmetriesVoltageOperations_preprint}. In joint work with Hubard and Mochán we developed a technique to describe classical geometrical operations of polyhedra, maps, polytopes and similar objects in a combinatorial way. This work lead to the publication of a paper in Combinatorica in 2023 \cite{HubaMocMon2023_VoltageOperationsManiplexes} and a second paper \cite{HubaMocMon_SymmetriesVoltageOperations_preprint} on the topic is currently under review.
	I am confident that our work has potential to become a breakthrough in the theory of highly symmetric objects and it has already caught the attention of the community.
  
	\item \textbf{Classification of highly symmetric tilings of the $n$-dimensional torus} \cite{ColliMonte2021_EquivelarToroidsFew}. Highly symmetric tessellations of the $2$-dimensional torus were described by Coxeter in the 70's. Cubic tilings (regardless of their symmetry degree) of the $3$-dimensional torus were classified in 2012 by Hubard, et. al. \cite{HubOrbPeWe2012_SymmetriesEquivelar4} and they conjectured the non-existence of tilings of a particular symmetry class on higher dimensions. 
  In a paper with Collins \cite{ColliMonte2021_EquivelarToroidsFew} we disproved this conjecture and completely classified the highly symmetric cubic tilings of the $n$-dimensional torus for arbitrary $n$. 
	\item \textbf{Classification of toroidal regular polyhedra} \cite{Montero2018_RegularPolyhedra3}. Regular polyhedra are classical objects whose study lies in the interplay of combinatorics and geometry. 
  The modern definition of regular polyhedron and a large list of examples were provided by Grünbaum in the 70's and in the 80's Dress achieved a complete classification of the regular polyhedra in the Euclidean space. A natural family of problems arise when we change the ambient space: the analogous problem for the projective space was attacked and eventually solved in the early 2000's while the hyperbolic analogous remains mostly open. In my first paper \cite{Montero2018_RegularPolyhedra3} I complete the classification of regular polyhedra in the $3$-dimensional torus.
  
\end{enumerate}

% [Provide a list of up to ten research outputs that demonstrate how you have advanced knowledge in your field with an emphasis on more recent achievements, such as publications, articles deposited in a publicly available preprint server, books, book chapters, conference proceedings, data sets, software, patents, licenses, standards, start-up businesses or any other research outputs you deem relevant in relation to your research field and your project.

% You may include a short, factual explanation of the significance of the selected outputs, your role in producing each of them, and how they demonstrate your capacity to successfully carry out your proposed project.]

\subsubsection*{Peer recognition}

\cventry{%
{January 2021}%
%%\es{Ene. 2021 -- Dic. 2024}%
}
{
Candidato a investigador nacional%
{ (National research candidate)}%
}%
{%
{National Council of Humanities, Science and Technology (CONAHCyT)}%
%%\es{Consejo Nacional de Humanidades, Ciencia y Tecnología (CONAHCyT)}%
}
{%
{Mexico.}%
%%\es{México.}%
}
{}
{
{As part of the program \emph{Sistema nacional de investigadores (SNI) (National researchers system)}.}%
%%\es{Como parte del programa \emph{Sistema nacional de investigadores e investigadoras (SNII)}.}
}
%    \cventry{Feb. 2015 - July 2015}{PAPIIT project IN101615 ``Politopos altamente simétricos en espacios de dimensión tres y cuatro''}{UNAM}{}{}{Grant given while writing my graduate thesis.}
%    \cventry{Feb. 2013 - Jan. 2015}{Grant 353994 of the National Grants Program}{Consejo Nacional de Ciencia y Tecnología}{}{}{Grant given while studying on graduate program.}
%    \cventry{May 2012 - Dec. 2013}{PAPIIT project IN112512 ``Politopos altamente simétricos en espacios de dimensión pequeña''}{UNAM}{}{}{Grant given while writing my undergraduate thesis.}
\cventry{
{July 2018}%
%%%\es{Julio 2018}%
}
{
{Best student talk award}%
% %%\es{Mejor plática por un estudiante}%
}
{UP FAMNIT - UP IAM - IMFM}
{
{Slovenia.}%
% %%\es{Eslovenia.}%
}
{}
{
{Award given to the best talk presented by a student during the 8th PhD Summer School in Discrete Mathematics.}%
% %%\es{Premio dado a la mejor plática presentada por un estudiante en la escuela 8th PhD Summer School in Discrete Mathematics.}
}

\cventry{
{October 2012}%
% %%\es{Octubre 2012}%
}
{``Padre de la Patria''}
{
{Faculty of Physics and Mathematics, UMSNH}%
% %%\es{Facultad de Ciencias Físico Matemáticas, UMSNH}%
}
{
{Mexico.}%
% %%\es{México.}%
}
{}
{
{Award given to the best student of each year.}%
% %%\es{Premio dado al mejor estudiante de la generación.}%
}

\cventry{
{October 2011}%
% %%\es{Octubre 2011}%
}
{``Padre de la Patria''}
{
{Faculty of Physics and Mathematics, UMSNH}%
% %%\es{Facultad de Ciencias Físico Matemáticas, UMSNH}%
}
{
{Mexico.}%
% %%\es{México.}%
}
{}
{
{Award given to the best student of each year.}%
% %%\es{Premio dado al mejor estudiante de la generación.}%
}


% [Provide a list of selected examples of significant recognition by your peers if applicable, such as prizes, awards, fellowships, elected academy memberships, invited presentations to major conferences or any other examples of significant recognition you deem relevant in relation to your research field and project.

% You may include a short explanation of the significance of the listed examples.]

% \subsection*{\underline{ADDITIONAL INFORMATION}}


% You may provide relevant additional information on your research career to provide context to the evaluation panels when assessing your research achievements and peer recognition as described above.

% \subsubsection*{Career breaks, diverse career paths and major life events}

% [You may include a short factual explanation of career breaks or diverse career paths such as secondments, volunteering, part-time work, time spent in different sectors or the effects of major life events such as long term illness as well as the effects of pandemic restrictions on research productivity.]

\subsubsection*{Other contributions to the research community} 

\paragraph{Journal Reviewer}
  \begin{itemize}
		\item{Ars Mathematica Contemporanea}
		\item{Discrete Mathematics}
		\item{Contributions to Discrete Mathematics}
		\item{The Art of Discrete and Applied Mathematics}
		\item{Journal of the Australian Mathematical Society}
  \end{itemize}

  \paragraph{Organising commitees}\leavevmode

  \cventry{
	  {June 2023}%
	  %\es{Junio 2023}%
	  }
	  {
	  {10th Slovenian Conference on Graph Theory}%
	  }
	  {}
	  {
	  {Kranjska Gora, Slovenia.}%
	  %\es{Kranjska Gora, Eslovenia.}%
	  }
	  {}
	  {
	  {Member of the organising committee.}%
	  %\es{Miembro del comité organizador.}%
	  }

  \cventry{
	  {May 2021}%
	  %\es{May. 2021 - Nov. 2021}%
	  }
	  {
	  {Mexican Mathematical Olympiad in Michoacan}%
	  %\es{Olimpiada mexicana de matemáticas en Michoacán}%
	  }
	  {}
	  {
	  {Morelia, Mich., Mexico.}%
	  %\es{Morelia, Mich., México.}%
	  }
	  {}
	  {
	  {Member of the local committee.}%
	  %\es{Miembro del comité local.}%
	  }

    \cventry{
    {February 2018}%
    %\es{Febrero 2018}%
    }
    {Degustaciones Matemáticas 2018}
    {
    {Joint Program of Graduate Studies in Mathematics UMSNH-UNAM}%
    %\es{Posgrado Conjunto en Ciencias Matemáticas UMSNH-UNAM}%
    }
    {
    {Morelia, Mich., Mexico.}%
    %\es{Morelia, Mich., México.}%
    }
    {}
    {
    {Cycle of talks in the Faculty of Physics and Mathematics, UMSNH.}%
    %\es{Ciclo de conferencias en la Facultad de Ciencias Fisico-Matemáticas UMSNH.}%
    }

    \cventry{
    {February 2017}%
    %\es{Febrero 2017}%
    }
    {Degustaciones Matemáticas 2017}
    {
    {Joint Program of Graduate Studies in Mathematics UMSNH-UNAM}%
    %\es{Posgrado Conjunto en Ciencias Matemáticas UMSNH-UNAM}%
    }
    {
    {Morelia, Mich., Mexico.}%
    %\es{Morelia, Mich., México.}%
    }
    {}
    {
    {Cycle of talks in the Faculty of Physics and Mathematics, UMSNH.}%
    %\es{Ciclo de conferencias en la Facultad de Ciencias Fisico-Matemáticas UMSNH.}%
    }

    \cventry{
    {February 2016}%
    %\es{Febrero 2016}%
    }
    {Degustaciones Matemáticas 2016}
    {
    {Joint Program of Graduate Studies in Mathematics UMSNH-UNAM}%
    %\es{Posgrado Conjunto en Ciencias Matemáticas UMSNH-UNAM}%
    }
    {
    {Morelia, Mich., Mexico.}%
    %\es{Morelia, Mich., México.}%
    }
    {}
    {
    {Cycle of talks in the Faculty of Physics and Mathematics, UMSNH.}%
    %\es{Ciclo de conferencias en la Facultad de Ciencias Fisico-Matemáticas UMSNH.}%
    }

    \cventry{
    {February 2015}%
    %\es{Febrero 2015}%
    }
    {Degustaciones Matemáticas 2015}
    {
    {Joint Program of Graduate Studies in Mathematics UMSNH-UNAM}%
    %\es{Posgrado Conjunto en Ciencias Matemáticas UMSNH-UNAM}%
    }
    {
    {Morelia, Mich., Mexico.}%
    %\es{Morelia, Mich., México.}%
    }
    {}
    {
    {Cycle of talks in the Faculty of Physics and Mathematics, UMSNH.}%
    %\es{Ciclo de conferencias en la Facultad de Ciencias Fisico-Matemáticas UMSNH.}%
    }



    \cventry{
    {Aug. 2014}%
    %\es{Ago. 2014 - Jun. 2015}%
    }
    {
    {Graduate students seminar}%
    %\es{Seminario de estudiantes de posgrado}%
    }
    {
    {Joint Program of Graduate Studies in Mathematics UMSNH-UNAM}%
    %\es{Posgrado Conjunto en Ciencias Matemáticas UMSNH-UNAM}%
    }
    {
    {Morelia, Mich., Mexico.}
    %\es{Morelia, Mich., México.}
    }
    {}
    {}

    \cventry{
    {March 2013}%
    %\es{Marzo 2013}%
    }
    {
    {XXVIII Colloquium \textit{``Victor Neumann-Lara''} of graph theory, combinatorics and its applications}%
    %\es{XXVIII Coloquio \textit{``Victor Neumann-Lara''} de teoría de las gráficas, combinatoria y sus aplicaciones}%
    }
    {}
    {
    {Morelia, Mich., Mexico.}%
    %\es{Morelia, Mich., México.}%
    }
    {}
    {
    {Member of the local committee.}%
    %\es{Miembro del comité local.}%
    }

    

% [You may include a list of particularly noteworthy contributions to the research community you have made other than research achievements and peer recognition and a short explanation of these contributions. The purpose of this section is to allow the panels to take a more rounded view of your career and achievements and to ensure that any additional responsibilities, commitments and leadership roles that you have taken on beyond your individual research activities are recognised and taken into account.]


% [(for more information see ‘Information for Applicants to the Starting and Consolidator Grant 2025 Calls’)]
